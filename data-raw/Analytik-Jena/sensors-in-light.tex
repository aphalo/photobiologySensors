% Options for packages loaded elsewhere
\PassOptionsToPackage{unicode}{hyperref}
\PassOptionsToPackage{hyphens}{url}
%
\documentclass[
]{article}
\usepackage{amsmath,amssymb}
\usepackage{lmodern}
\usepackage{iftex}
\ifPDFTeX
  \usepackage[T1]{fontenc}
  \usepackage[utf8]{inputenc}
  \usepackage{textcomp} % provide euro and other symbols
\else % if luatex or xetex
  \usepackage{unicode-math}
  \defaultfontfeatures{Scale=MatchLowercase}
  \defaultfontfeatures[\rmfamily]{Ligatures=TeX,Scale=1}
\fi
% Use upquote if available, for straight quotes in verbatim environments
\IfFileExists{upquote.sty}{\usepackage{upquote}}{}
\IfFileExists{microtype.sty}{% use microtype if available
  \usepackage[]{microtype}
  \UseMicrotypeSet[protrusion]{basicmath} % disable protrusion for tt fonts
}{}
\makeatletter
\@ifundefined{KOMAClassName}{% if non-KOMA class
  \IfFileExists{parskip.sty}{%
    \usepackage{parskip}
  }{% else
    \setlength{\parindent}{0pt}
    \setlength{\parskip}{6pt plus 2pt minus 1pt}}
}{% if KOMA class
  \KOMAoptions{parskip=half}}
\makeatother
\usepackage{xcolor}
\usepackage[margin=1in]{geometry}
\usepackage{color}
\usepackage{fancyvrb}
\newcommand{\VerbBar}{|}
\newcommand{\VERB}{\Verb[commandchars=\\\{\}]}
\DefineVerbatimEnvironment{Highlighting}{Verbatim}{commandchars=\\\{\}}
% Add ',fontsize=\small' for more characters per line
\usepackage{framed}
\definecolor{shadecolor}{RGB}{248,248,248}
\newenvironment{Shaded}{\begin{snugshade}}{\end{snugshade}}
\newcommand{\AlertTok}[1]{\textcolor[rgb]{0.94,0.16,0.16}{#1}}
\newcommand{\AnnotationTok}[1]{\textcolor[rgb]{0.56,0.35,0.01}{\textbf{\textit{#1}}}}
\newcommand{\AttributeTok}[1]{\textcolor[rgb]{0.77,0.63,0.00}{#1}}
\newcommand{\BaseNTok}[1]{\textcolor[rgb]{0.00,0.00,0.81}{#1}}
\newcommand{\BuiltInTok}[1]{#1}
\newcommand{\CharTok}[1]{\textcolor[rgb]{0.31,0.60,0.02}{#1}}
\newcommand{\CommentTok}[1]{\textcolor[rgb]{0.56,0.35,0.01}{\textit{#1}}}
\newcommand{\CommentVarTok}[1]{\textcolor[rgb]{0.56,0.35,0.01}{\textbf{\textit{#1}}}}
\newcommand{\ConstantTok}[1]{\textcolor[rgb]{0.00,0.00,0.00}{#1}}
\newcommand{\ControlFlowTok}[1]{\textcolor[rgb]{0.13,0.29,0.53}{\textbf{#1}}}
\newcommand{\DataTypeTok}[1]{\textcolor[rgb]{0.13,0.29,0.53}{#1}}
\newcommand{\DecValTok}[1]{\textcolor[rgb]{0.00,0.00,0.81}{#1}}
\newcommand{\DocumentationTok}[1]{\textcolor[rgb]{0.56,0.35,0.01}{\textbf{\textit{#1}}}}
\newcommand{\ErrorTok}[1]{\textcolor[rgb]{0.64,0.00,0.00}{\textbf{#1}}}
\newcommand{\ExtensionTok}[1]{#1}
\newcommand{\FloatTok}[1]{\textcolor[rgb]{0.00,0.00,0.81}{#1}}
\newcommand{\FunctionTok}[1]{\textcolor[rgb]{0.00,0.00,0.00}{#1}}
\newcommand{\ImportTok}[1]{#1}
\newcommand{\InformationTok}[1]{\textcolor[rgb]{0.56,0.35,0.01}{\textbf{\textit{#1}}}}
\newcommand{\KeywordTok}[1]{\textcolor[rgb]{0.13,0.29,0.53}{\textbf{#1}}}
\newcommand{\NormalTok}[1]{#1}
\newcommand{\OperatorTok}[1]{\textcolor[rgb]{0.81,0.36,0.00}{\textbf{#1}}}
\newcommand{\OtherTok}[1]{\textcolor[rgb]{0.56,0.35,0.01}{#1}}
\newcommand{\PreprocessorTok}[1]{\textcolor[rgb]{0.56,0.35,0.01}{\textit{#1}}}
\newcommand{\RegionMarkerTok}[1]{#1}
\newcommand{\SpecialCharTok}[1]{\textcolor[rgb]{0.00,0.00,0.00}{#1}}
\newcommand{\SpecialStringTok}[1]{\textcolor[rgb]{0.31,0.60,0.02}{#1}}
\newcommand{\StringTok}[1]{\textcolor[rgb]{0.31,0.60,0.02}{#1}}
\newcommand{\VariableTok}[1]{\textcolor[rgb]{0.00,0.00,0.00}{#1}}
\newcommand{\VerbatimStringTok}[1]{\textcolor[rgb]{0.31,0.60,0.02}{#1}}
\newcommand{\WarningTok}[1]{\textcolor[rgb]{0.56,0.35,0.01}{\textbf{\textit{#1}}}}
\usepackage{graphicx}
\makeatletter
\def\maxwidth{\ifdim\Gin@nat@width>\linewidth\linewidth\else\Gin@nat@width\fi}
\def\maxheight{\ifdim\Gin@nat@height>\textheight\textheight\else\Gin@nat@height\fi}
\makeatother
% Scale images if necessary, so that they will not overflow the page
% margins by default, and it is still possible to overwrite the defaults
% using explicit options in \includegraphics[width, height, ...]{}
\setkeys{Gin}{width=\maxwidth,height=\maxheight,keepaspectratio}
% Set default figure placement to htbp
\makeatletter
\def\fps@figure{htbp}
\makeatother
\setlength{\emergencystretch}{3em} % prevent overfull lines
\providecommand{\tightlist}{%
  \setlength{\itemsep}{0pt}\setlength{\parskip}{0pt}}
\setcounter{secnumdepth}{-\maxdimen} % remove section numbering
\ifLuaTeX
  \usepackage{selnolig}  % disable illegal ligatures
\fi
\IfFileExists{bookmark.sty}{\usepackage{bookmark}}{\usepackage{hyperref}}
\IfFileExists{xurl.sty}{\usepackage{xurl}}{} % add URL line breaks if available
\urlstyle{same} % disable monospaced font for URLs
\hypersetup{
  pdftitle={Analityk Jena sensors},
  pdfauthor={Pedro J. Aphalo},
  hidelinks,
  pdfcreator={LaTeX via pandoc}}

\title{Analityk Jena sensors}
\author{Pedro J. Aphalo}
\date{2022-12-07}

\begin{document}
\maketitle

\begin{Shaded}
\begin{Highlighting}[]
\FunctionTok{library}\NormalTok{(photobiologySensors)}
\end{Highlighting}
\end{Shaded}

\begin{verbatim}
## Loading required package: photobiology
\end{verbatim}

\begin{verbatim}
## News at https://www.r4photobiology.info/
\end{verbatim}

\begin{Shaded}
\begin{Highlighting}[]
\FunctionTok{library}\NormalTok{(photobiologyLamps)}
\FunctionTok{library}\NormalTok{(photobiologyFilters)}

\FunctionTok{library}\NormalTok{(ggspectra)}
\end{Highlighting}
\end{Shaded}

\begin{verbatim}
## Loading required package: ggplot2
\end{verbatim}

\begin{Shaded}
\begin{Highlighting}[]
\FunctionTok{photon\_as\_default}\NormalTok{()}
\end{Highlighting}
\end{Shaded}

\hypertarget{sensors-on-their-own}{%
\subsection{Sensors on their own}\label{sensors-on-their-own}}

Digitized from manual assuming that energy response is shown, and
plotted as photon response normalized to the wavelength of maximum
sensitivity.

\begin{Shaded}
\begin{Highlighting}[]
\FunctionTok{autoplot}\NormalTok{(sensors.mspct[}\FunctionTok{c}\NormalTok{(}\StringTok{"Analytik\_Jena\_UVX31"}\NormalTok{, }\StringTok{"Analytik\_Jena\_UVX36"}\NormalTok{)],}
         \AttributeTok{facets =} \DecValTok{1}\NormalTok{) }\SpecialCharTok{+}
  \FunctionTok{ggtitle}\NormalTok{(}\StringTok{"Spectral photon response"}\NormalTok{)}
\end{Highlighting}
\end{Shaded}

\includegraphics{sensors-in-light_files/figure-latex/unnamed-chunk-2-1.pdf}

\hypertarget{sensors-in-sunlight}{%
\subsection{Sensors in sunlight}\label{sensors-in-sunlight}}

By combining the sensor response with a photon irradiance spectrum of
sunlight for mid morning in Helsinki, we see that UVX-36 might work with
a sunlight especific calibration but the UVX-31 could be calibrated for
an approximate measurement of UV-A but is useless for UV-B in sunlight.

\begin{Shaded}
\begin{Highlighting}[]
\FunctionTok{autoplot}\NormalTok{(}\FunctionTok{convolve\_each}\NormalTok{(sensors.mspct[}\FunctionTok{c}\NormalTok{(}\StringTok{"Analytik\_Jena\_UVX31"}\NormalTok{, }\StringTok{"Analytik\_Jena\_UVX36"}\NormalTok{)], sun.spct),}
         \AttributeTok{facets =} \DecValTok{1}\NormalTok{) }\SpecialCharTok{+}
  \FunctionTok{ggtitle}\NormalTok{(}\StringTok{"Response to sunlight"}\NormalTok{)}
\end{Highlighting}
\end{Shaded}

\includegraphics{sensors-in-light_files/figure-latex/unnamed-chunk-3-1.pdf}

\hypertarget{sensors-with-broad-band-uv-b-lamps}{%
\subsection{Sensors with broad-band UV-B
lamps}\label{sensors-with-broad-band-uv-b-lamps}}

As we can expect that the spectrum of the radiation from the lamps
varies little or not a all, the sensors could be calibrated to be
useful. Anyway, they would require separate calibrations for bare lamps,
lamps filtered with cellulose diacetate and lamps filtered with
polyester.

\begin{Shaded}
\begin{Highlighting}[]
\FunctionTok{autoplot}\NormalTok{(}\FunctionTok{convolve\_each}\NormalTok{(sensors.mspct[}\FunctionTok{c}\NormalTok{(}\StringTok{"Analytik\_Jena\_UVX31"}\NormalTok{, }\StringTok{"Analytik\_Jena\_UVX36"}\NormalTok{)],}
\NormalTok{                       lamps.mspct}\SpecialCharTok{$}\NormalTok{Philips.FT.TL}\FloatTok{.40}\NormalTok{W.}\FloatTok{12.}\NormalTok{uv),}
         \AttributeTok{facets =} \DecValTok{1}\NormalTok{) }\SpecialCharTok{+}
  \FunctionTok{ggtitle}\NormalTok{(}\StringTok{"Response to Philips TL 12 (UV{-}B broad)"}\NormalTok{)}
\end{Highlighting}
\end{Shaded}

\includegraphics{sensors-in-light_files/figure-latex/unnamed-chunk-4-1.pdf}

\begin{Shaded}
\begin{Highlighting}[]
\FunctionTok{autoplot}\NormalTok{(}\FunctionTok{convolve\_each}\NormalTok{(sensors.mspct[}\FunctionTok{c}\NormalTok{(}\StringTok{"Analytik\_Jena\_UVX31"}\NormalTok{, }\StringTok{"Analytik\_Jena\_UVX36"}\NormalTok{)],}
\NormalTok{                       (lamps.mspct}\SpecialCharTok{$}\NormalTok{Philips.FT.TL}\FloatTok{.40}\NormalTok{W.}\FloatTok{12.}\NormalTok{uv }\SpecialCharTok{*}\NormalTok{ filters.mspct}\SpecialCharTok{$}\NormalTok{Courtaulds\_CA\_115um)),}
         \AttributeTok{facets =} \DecValTok{1}\NormalTok{) }\SpecialCharTok{+}
  \FunctionTok{ggtitle}\NormalTok{(}\StringTok{"Response to Philips TL 12 (UV{-}B broad) + acetate"}\NormalTok{)}
\end{Highlighting}
\end{Shaded}

\includegraphics{sensors-in-light_files/figure-latex/unnamed-chunk-5-1.pdf}

\begin{Shaded}
\begin{Highlighting}[]
\FunctionTok{autoplot}\NormalTok{(}\FunctionTok{convolve\_each}\NormalTok{(sensors.mspct[}\FunctionTok{c}\NormalTok{(}\StringTok{"Analytik\_Jena\_UVX31"}\NormalTok{, }\StringTok{"Analytik\_Jena\_UVX36"}\NormalTok{)],}
\NormalTok{                       (lamps.mspct}\SpecialCharTok{$}\NormalTok{Philips.FT.TL}\FloatTok{.40}\NormalTok{W.}\FloatTok{12.}\NormalTok{uv }\SpecialCharTok{*}\NormalTok{ polyester.spct)),}
         \AttributeTok{facets =} \DecValTok{1}\NormalTok{) }\SpecialCharTok{+}
  \FunctionTok{ggtitle}\NormalTok{(}\StringTok{"Response to Philips TL 12 (UV{-}B broad) + PET"}\NormalTok{)}
\end{Highlighting}
\end{Shaded}

\includegraphics{sensors-in-light_files/figure-latex/unnamed-chunk-6-1.pdf}

\hypertarget{sensors-with-narrow-band-uv-b-lamps}{%
\subsection{Sensors with narrow-band UV-B
lamps}\label{sensors-with-narrow-band-uv-b-lamps}}

As we can expect that the spectrum of the radiation from the lamps
varies little or not a all, the sensors could be calibrated to be
useful. Anyway, they would require separate calibrations for bare lamps
and lamps filtered with polyester.

\begin{Shaded}
\begin{Highlighting}[]
\FunctionTok{autoplot}\NormalTok{(}\FunctionTok{convolve\_each}\NormalTok{(sensors.mspct[}\FunctionTok{c}\NormalTok{(}\StringTok{"Analytik\_Jena\_UVX31"}\NormalTok{, }\StringTok{"Analytik\_Jena\_UVX36"}\NormalTok{)],}
\NormalTok{                       lamps.mspct}\SpecialCharTok{$}\NormalTok{Philips.FT.TL}\FloatTok{.40}\NormalTok{W.}\FloatTok{01.}\NormalTok{uv),}
         \AttributeTok{facets =} \DecValTok{1}\NormalTok{) }\SpecialCharTok{+}
  \FunctionTok{ggtitle}\NormalTok{(}\StringTok{"Response to Philips TL 01 (UV{-}B narrow)"}\NormalTok{)}
\end{Highlighting}
\end{Shaded}

\includegraphics{sensors-in-light_files/figure-latex/unnamed-chunk-7-1.pdf}

\begin{Shaded}
\begin{Highlighting}[]
\FunctionTok{autoplot}\NormalTok{(}\FunctionTok{convolve\_each}\NormalTok{(sensors.mspct[}\FunctionTok{c}\NormalTok{(}\StringTok{"Analytik\_Jena\_UVX31"}\NormalTok{, }\StringTok{"Analytik\_Jena\_UVX36"}\NormalTok{)],}
\NormalTok{                       (lamps.mspct}\SpecialCharTok{$}\NormalTok{Philips.FT.TL}\FloatTok{.40}\NormalTok{W.}\FloatTok{01.}\NormalTok{uv }\SpecialCharTok{*}\NormalTok{ filters.mspct}\SpecialCharTok{$}\NormalTok{Courtaulds\_CA\_115um)),}
         \AttributeTok{facets =} \DecValTok{1}\NormalTok{) }\SpecialCharTok{+}
  \FunctionTok{ggtitle}\NormalTok{(}\StringTok{"Response to Philips TL 01 (UV{-}B narrow) + acetate"}\NormalTok{)}
\end{Highlighting}
\end{Shaded}

\includegraphics{sensors-in-light_files/figure-latex/unnamed-chunk-8-1.pdf}

\begin{Shaded}
\begin{Highlighting}[]
\FunctionTok{autoplot}\NormalTok{(}\FunctionTok{convolve\_each}\NormalTok{(sensors.mspct[}\FunctionTok{c}\NormalTok{(}\StringTok{"Analytik\_Jena\_UVX31"}\NormalTok{, }\StringTok{"Analytik\_Jena\_UVX36"}\NormalTok{)],}
\NormalTok{                       (lamps.mspct}\SpecialCharTok{$}\NormalTok{Philips.FT.TL}\FloatTok{.40}\NormalTok{W.}\FloatTok{01.}\NormalTok{uv }\SpecialCharTok{*}\NormalTok{ polyester.spct)),}
         \AttributeTok{facets =} \DecValTok{1}\NormalTok{) }\SpecialCharTok{+}
  \FunctionTok{ggtitle}\NormalTok{(}\StringTok{"Response to Philips TL 01 (UV{-}B narrow) + PET"}\NormalTok{)}
\end{Highlighting}
\end{Shaded}

\includegraphics{sensors-in-light_files/figure-latex/unnamed-chunk-9-1.pdf}

\end{document}
